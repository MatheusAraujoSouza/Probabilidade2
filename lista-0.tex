%% LICENSE: https://creativecommons.org/licenses/by-sa/4.0/legalcode


%% Este arquivo pode ser modificado e reproduzido por qualquer meio,
%% para qualquer fim, desde que mantidos os autores e editores da
%% lista abaixo. E o código gerado seja disponível de forma pública.
%% conforme os critérios da licença Creative Commons, versão
%% Attribution-ShareAlike (https://creativecommons.org/licenses/by-sa/4.0/legalcode).
%% Peço somente (não há obrigatoriedade) que caso faça alguma alteração,
%% me envie as alterações para que possamos melhorar este documento.
%% Obrigado.

%% Author: Eric Lopes - https://github.com/nullhack/Probabilidade2 - 02/09/2012
%% Edited by: Eric Lopes - https://github.com/nullhack/Probabilidade2 - 14/09/2015
%% Edited by: Eric Lopes - https://github.com/nullhack/Probabilidade2 - 24/07/2016
%% Edited by:

\documentclass[portuguese]{article}
\usepackage[T1]{fontenc}
\usepackage[utf8]{inputenc}
\usepackage{amsmath}
\usepackage{amssymb}
\usepackage{esint}
\usepackage{babel}
\begin{document}

\title{Probabilidade 2 - ME310 - Lista 0}

\maketitle

\subsubsection*{Lembrando:}


\subsubsection*{1) Conjuntos disjuntos: $A\cap B=\emptyset\protect\implies P(A\cap B)=0$}


\subsubsection*{2) Conjuntos independentes: $P(A\cap B)=P(A)\cdot P(B)$}


\subsubsection*{3) Podemos dividir qualquer conjunto em dois conjuntos disjuntos:
$A=(A\cap B)\protect\overset{.}{\cup}(A\cap B^{c})$}


\subsubsection*{4) Notação para união disjunta: $A\protect\overset{.}{\cup}B$ é
só uma forma de deixar explicito que $A\cap B=\emptyset$ e que estamos
fazendo uma união $A\protect\overset{}{\cup}B$ }


\subsubsection*{5) Probabilidade de união de conjuntos $P(A\cup B)\protect\overset{}{=}P(A)+P(B)-P(A\cap B)$}


\subsubsection*{6) Probabilidade condicional: $P(A/B)=\frac{P(A\cap B)}{P(B)}$}


\subsubsection*{7) Fórmula de Bayes: $P(K)=P(K\cap W)+P(K\cap W^{c})=P(K/W)\cdot P(W)+P(K/W^{c})\cdot P(W^{c})\protect\implies P(K/W)=\frac{P(W/K)\cdot P(K)}{P(W/K)\cdot P(K)+P(W/K^{c})\cdot P(K^{c})}$}


\subsubsection*{8) $P(\Omega)=1$ e $P(\emptyset)=0$}


\subsubsection*{9) $X\sim Poisson(\lambda)\protect\implies f(x)=e^{-\lambda}\cdot\frac{\lambda^{x}}{x!}$
e $\mathbf{E}(X)=\lambda$}


\subsubsection*{10) $X\sim Exp(\lambda)\protect\implies f(x)=\lambda\cdot e^{-\lambda\cdot x}$
e $\mathbf{E}(X)=\frac{1}{\lambda}$}


\subsubsection*{11) $X\sim Uniforme(a,b)\protect\implies f(x)=\frac{1}{b-a}$ e $F_{X}(x)=\protect\begin{cases}
\protect\begin{array}{c}
\frac{x-a}{b-a}\protect\\
1
\protect\end{array} & \protect\begin{array}{c}
a\leq x\leq b\protect\\
x>b
\protect\end{array}\protect\end{cases}$ e $\mathbf{E}(X)=\frac{a+b}{2}$}

\textcompwordmark{}


\subsection*{\textmd{1) Sejam A e B eventos disjuntos tais que P(A) = 0,1 e P(B)=0,4
. Qual é a probabilidade que:}}


\subsubsection*{\textmd{a) A ou B ocorra e são disjuntos}}


\subsubsection*{\textmd{Resp. a) $P(A\cup B)\protect\overset{}{=}P(A)+P(B)-P(A\cap B)\protect\overset{1}{=}P(A)+P(B)-0=0,1+0,4$}}


\subsubsection*{\textmd{b) A ocorra mas não B e são disjuntos}}


\subsubsection*{\textmd{Resp. b) $P(A\cap B^{c})\protect\overset{3}{=}P(A)-P(A\cap B)\protect\overset{1}{=}P(A)-0$}}


\subsubsection*{\textmd{c) A ou B ocorra e são independentes}}


\subsubsection*{\textmd{Resp. c)$P(A\cup B)\protect\overset{}{=}P(A)+P(B)-P(A\cap B)\protect\overset{2}{=}P(A)+P(B)-P(A)\cdot P(B)=0,1+0,4-(0,1\cdot0,4)$}}


\subsubsection*{\textmd{d) A ocorra mas não B e são independentes}}


\subsubsection*{\textmd{Resp. d) $P(A\cap B^{c})\protect\overset{1,3,5}{=}P(A)-P(A\cap B)\protect\overset{2}{=}P(A)-P(A)\cdot P(B)=0,1-(0,1\cdot0,4)$}}

\textbf{\textcompwordmark{}}


\subsection*{\textmd{2) Considere duas urnas, a urna A e a urna B. Urna A contém
4 bolas vermelhas, 3 bolas azuis e 2 bolas verdes. A urna B contém
2 bolas vermelhas, 3 bolas azuis e 4 bolas verdes. Uma bola é retirada
da urna A e colocada na urna B. Depois, uma bola é retirada da urna
B.}}


\subsubsection*{\textmd{Para simplificar, vamos considerar que V = verde; Z = aZul;
D = verDe}}


\subsubsection*{\textmd{A notação que eu vou usar é $X_{y}$ em que X representa
a cor que nos interessa e Y a urna de interesse}}


\subsubsection*{\textmd{Pelo enunciado temos $P(V_{A})=4/9$; $P(Z_{A})=3/9$; $P(D_{A})=2/9$. }}


\subsubsection*{\textmd{a) Qual a probabilidade de que uma bola retirada da urna
B seja vermelha ?}}


\subsubsection*{\textmd{Resp. a) $P(V_{B})\protect\overset{7}{=}P(V_{B}/V_{A})\cdot P(V_{A})+P(V_{B}/V_{A}^{c})\cdot P(V_{A}^{c})=\frac{3}{10}\cdot\frac{4}{9}+\frac{2}{10}\cdot\frac{5}{9}=\frac{22}{90}$}}


\subsubsection*{\textmd{b) Se uma bola vermelha é retirada da urna B, qual é a probabilidade
de que uma bola vermelha tenha sido retirada da urna A ?}}


\subsubsection*{\textmd{Resp. b) $P(V_{A}/V_{B})\protect\overset{6}{=}\frac{P(V_{A}\cap V_{B})}{P(V_{B})}=\frac{P(V_{B}\cap V_{A})}{P(V_{B})}\protect\overset{6}{=}P(V_{B}/V_{A})\cdot\frac{P(V_{A})}{P(V_{B})}=\frac{3}{10}\cdot\frac{4/9}{22/90}=\frac{12}{22}$}}

\textcompwordmark{}


\subsection*{\textmd{3) Demonstre as seguintes afirmações:}}


\subsubsection*{\textmd{a) Se $P(A)=0$ e B é um evento qualquer, então A e B são
independentes:}}


\subsubsection*{\textmd{Resp. a) $A\cap B\subset A\protect\implies0\le P(A\cap B)\le P(A)=0$
Logo $P(A\cap B)=0=0\cdot P(B)=P(A)\cdot P(B)$}}


\subsubsection*{\textmd{b) Se $P(A)=1$ e B é um evento qualquer, então A e B são
independentes:}}


\subsubsection*{\textmd{Resp. b) $P(A^{c})=0$ e $P(B)=P(B\cap A)+P(B\cap A^{c})=P(B\cap A)$
Logo $P(B\cap A)=P(B)=1\cdot P(B)=P(A)\cdot P(B)$}}


\subsubsection*{\textmd{c) Os eventos $D$ e $D^{c}$ são independentes se e somente
se $P(D)=0$ ou $P(D)=1$}}


\subsubsection*{\textmd{Resp. c) Se $P(D\cap D^{c})=P(\emptyset)=0=P(D)\cdot P(D^{c})$
Então $P(D^{c})=0$ ou $P(D)=0$ }}


\subsubsection*{\textmd{d) Ache uma condição para que o evento E seja independente
dele mesmo}}


\subsubsection*{\textmd{Resp. d) $P(E)=P(E\cap E)=P(E)\cdot P(E)\protect\implies P(E)=P(E)^{2}\protect\implies P(E)=0$
ou $P(E)=1$}}

\textbf{\textcompwordmark{}}


\subsection*{\textmd{4) Suponha que o número de vezes que uma pessoa fica resfriada
durante o um ano tem distribuição de poisson com parâmetro $\lambda=4$.
Um novo remédio pra prevenir resfriados reduz este parâmetro para
$\lambda'=2$ em $75\%$ das pessoas e não tem efeito nos $25\%$
restantes. Se uma pessoa tomou este remédio durante um ano e pegou
resfriado $2$ vezes, qual é a probabilidade de que o remédio funcione
para esta pessoa ?}}


\subsubsection*{\textmd{Minha notação: F = Funcionar ; N = Não funcionar; X = número
de resfriados durante o ano}}


\subsubsection*{\textmd{Do enunciado: $P(F)=0,75$; $P(N)=0,25$; $P(X=n/F)\sim Poisson(2)$;
$P(X=n/N)\sim Poisson(4)$}}


\subsubsection*{\textmd{Resp.) $P(F/X=2)=\frac{P(F\cap X=2)}{P(X=2)}=\frac{P(X=2\cap F)}{P(X=2)}\protect\overset{7}{=}\frac{P(X=2/F)\cdot P(F)}{P(X=2/F)\cdot P(F)+P(X=2/N)\cdot P(N)}=\frac{\frac{e^{-2}\cdot2^{2}}{2!}\cdot0,75}{\frac{e^{-2}\cdot2^{2}\cdot0,75}{2!}+\frac{e^{-2}\cdot4^{2}\cdot0,25}{2!}}$}}

\textbf{\textcompwordmark{}}


\subsection*{\textmd{5) Seja X a v.a. contínua cuja densidade de probabilidade
é $f(x)=\protect\begin{cases}
\protect\begin{array}{c}
k\cdot x^{3}\protect\\
0
\protect\end{array} & \protect\begin{array}{c}
0\leq x\leq1\protect\\
cc
\protect\end{array}\protect\end{cases}$}}


\subsubsection*{\textmd{a) Determine o valor de k}}


\subsubsection*{\textmd{Resp. a) $1=\int_{-\infty}^{\infty}f(x)dx=\int_{0}^{1}f(x)dx=\int_{0}^{1}k\cdot x^{3}dx\protect\implies k=4$}}


\subsubsection*{\textmd{b) Calcule $P(1/4<X<1/2)$}}


\subsubsection*{\textmd{Resp. b) $\int_{1/4}^{1/2}4\cdot x^{3}dx=\frac{1}{2^{4}}-\frac{1}{4^{4}}$}}


\subsubsection*{\textmd{c) Calcule $\mathbf{E}(X)$, $Var(X)$}}


\subsubsection*{\textmd{Resp. c) $\mathbf{E}(X)=\int_{-\infty}^{\infty}x\cdot f(x)dx=\int_{0}^{1}x\cdot4\cdot x^{3}dx=\int_{0}^{1}4\cdot x^{4}dx=\frac{4}{5}$}}


\subsubsection*{\textmd{$\mathbf{E}(X^{2})=\int_{-\infty}^{\infty}x{}^{2}\cdot f(x)dx=\int_{0}^{1}x^{2}\cdot4\cdot x^{3}dx=\int_{0}^{1}4\cdot x^{5}dx=\frac{4}{6}$}}

\textbf{$Var(x)=\mathbf{E}((X-\mu)^{2})=\mathbf{E}(X^{2})-\mathbf{E}(X)^{2}=\frac{4}{6}-(\frac{4}{5})^{2}$}


\subsubsection*{\textmd{d) Determine a f.d.a. de X}}


\subsubsection*{\textmd{Resp. d) $F(n)=\int_{-\infty}^{n}f(x)dx=\int_{0}^{n}4\cdot x^{3}dx=\protect\begin{cases}
\protect\begin{array}{c}
n^{4}\protect\\
0
\protect\end{array} & \protect\begin{array}{c}
0\leq n\leq1\protect\\
cc
\protect\end{array}\protect\end{cases}$}}

\textbf{\textcompwordmark{}}


\subsection*{\textmd{6) O tempo que um eletrodoméstico funciona (até quebrar)
tem distribuição exponencial com média 3 anos. Se uma pessoa comprou
um eletrodoméstico usado, calcule a probabilidade de que este vai
durar pelo menos mais 2 anos.}}


\subsubsection*{\textmd{Resp. ) $P(X\geq s+t/x\geq s)=\frac{P(X\geq s+t\cap X\geq s)}{P(X\geq s)}=\frac{P(X\geq s+t)}{P(x\geq s)}\protect\overset{10}{=}\frac{\int_{s+t}^{\infty}\lambda\cdot e^{-\lambda\cdot x}dx}{\int_{s}^{\infty}\lambda\cdot e^{-\lambda\cdot x}dx}=\frac{\lambda\cdot e^{-\lambda\cdot\left(s+t\right)}}{\lambda\cdot e^{-\lambda\cdot s}}=e^{-\lambda\cdot t}=\int_{t}^{\infty}\lambda\cdot e^{-\lambda\cdot x}dx=P(X\geq t)$}}

Do enunciado, temos que $\lambda=\frac{1}{3}$

$\implies P(X\geq s+2/x\geq s)=P(X\geq2)\overset{10}{=}\int_{2}^{\infty}\lambda\cdot e^{-\lambda\cdot x}dx=e^{-2\cdot\lambda}=e^{-\frac{2}{3}}$

\textbf{\textcompwordmark{}}


\subsection*{\textmd{7) Na fabricação de parafusos, os parafusos tem que ter diâmetro
entre $d_{1}$ e $d_{2}$, senão eles são considerados defeituosos.
Para controle de qualidade é feito um teste ``passa - não passa'',
o parafuso é aceito, se ele não passa numa abertura de diâmetro $d_{1}$,
mas passa numa abertura de diâmetro $d_{2}$. Suponha que o diâmetro
$D$ de um parafuso é uma v.a. Normal com média $\frac{\left(d_{1}+d_{2}\right)}{2}$
e variância $\frac{\left(d_{2}-d_{1}\right)^{2}}{16}$.}}


\subsubsection*{\textmd{Do enunciado: $d\in\left(d_{1},d_{2}\right)$ para passar
no teste, e o seu tamanho é uma variável aleatória $D\sim Normal(\frac{\left(d_{1}+d_{2}\right)}{2},\frac{\left(d_{2}-d_{1}\right)^{2}}{16})$,
estou usando $\epsilon\rightarrow0$}}


\subsubsection*{\textmd{a) Ache a probabilidade de um parafuso escolhido ao acaso
ser defeituoso}}


\subsubsection*{\textmd{Resp. a) $P(D<d_{1}\cup D>d_{2})=1-P(D\geq d_{1}\cup D\leq d_{2})=1-(F(d_{2})-F(d_{1}-\epsilon))=1+F(d_{1}-\epsilon)-F(d_{2})=1+\phi(\frac{d_{1}-\epsilon-\mu}{\sigma})-\phi(\frac{d_{2}-\mu}{\sigma})$}}


\subsubsection*{\textmd{b) Se em vez de saber a variância, você soubesse que $d_{1}=40mm$,
$d_{2}=50mm$ e que $10\%$ dos parafusos são rejeitados, quanto valeria
Var(D) ?}}


\subsubsection*{\textmd{Resp. b) $P(D<d_{1}\cup D>d_{2})=1-P(D\geq d_{1}\cup D\leq d_{2})=1-(F(d_{2})-F(d_{1}-\epsilon))=1+\phi(\frac{d_{1}-\epsilon-\mu}{\sigma})-\phi(\frac{d_{2}-\mu}{\sigma})=0,1$
e $\mu=45mm$ , já que }}

$F(a)=P(X\leq a)=P(\frac{x-\mu}{\sigma}\leq\frac{a-\mu}{\sigma})=\phi(\frac{a-\mu}{\sigma})$

disso temos que 

$1+\phi(\frac{d_{1}-\epsilon-\mu}{\sigma})-\phi(\frac{d_{2}-\mu}{\sigma})=0,1\implies\phi(\frac{-5-\epsilon}{\sigma})-\phi(\frac{5}{\sigma})=-0,9\implies1-\phi(\frac{5+\epsilon}{\sigma})-\phi(\frac{5}{\sigma})=-0,9\implies2\cdot\phi(\frac{5+\epsilon_{1}}{\sigma})=1,9\implies\phi(\frac{5+\epsilon_{1}}{\sigma})=0,95$

Olhando na tabela de distriuição normal temos $\phi(x)=0,95\implies x=1,64+\epsilon_{2}=\frac{5+\epsilon_{1}}{\sigma}\implies\sigma\backsimeq\frac{5}{1,64}$

\textcompwordmark{}


\subsection*{\textmd{8) Suponha que o raio R de uma esfera seja uma v.a. contínua
com densidade $f_{r}(r)=\protect\begin{cases}
\protect\begin{array}{c}
6\cdot r\cdot(1-r)\protect\\
0
\protect\end{array} & \protect\begin{array}{c}
0<r<1\protect\\
cc
\protect\end{array}\protect\end{cases}$, ache a densidade do volume V da esfera.}}


\subsubsection*{\textmd{Do nosso conhecimento sobre volume de sólidos temos que $V=\frac{4}{3}\cdot\pi\cdot R^{3}=k\cdot R^{3}$}}


\subsubsection*{\textmd{Resp. ) $F_{V}(v)=P(V\leq v)=P(k\cdot R^{3}\leq v)=P(R\leq\left(\frac{v}{k}\right)^{\frac{1}{3}})=\int_{0}^{\left(\frac{v}{k}\right)^{\frac{1}{3}}}6\cdot r\cdot(1-r)dr$
e temos a restrição $0\leq\left(\frac{v}{k}\right)^{\frac{1}{3}}\leq1\protect\implies v\in\left(0,\frac{4\cdot\pi}{3}\right)$}}

com isso temos

$F_{V}(v)=3\cdot\left(\frac{v}{k}\right)^{\frac{2}{3}}-2\cdot\frac{v}{k}$,
como queremos a densidade, temos que derivar em relação a v 

$f_{V}(v)=\frac{dF_{V}(v)}{dv}=2\cdot\left(\frac{v}{k}\right)^{\frac{-1}{3}}-\frac{2}{k}$
com $0<v<\frac{4\cdot\pi}{3}$

\textbf{\textcompwordmark{}}


\subsection*{\textmd{9) Seja $X$ uma v.a. com f.d.a. $F_{X}$}}


\subsubsection*{\textmd{a) Seja $Y=1+b\cdot X$. Ache a f.d.a. de Y (considere dois
casos: b>0 e b<0).}}


\subsubsection*{\textmd{Resp. a) Caso $b>0$: $P(Y\leq y)=P(1+b\cdot X\leq y)=P(X\leq\frac{y-1}{b})=F_{x}(\frac{y-1}{b})$}}


\subsubsection*{\textmd{Resp. a) Caso $b<0$: $P(Y\leq y)=P(1+b\cdot X\leq y)=P(X\geq\frac{y-1}{|b|})=1-F_{x}(\frac{y-1}{|b|})$}}


\subsubsection*{\textmd{b) Suponha que $F_{X}$ é estritamente monótona e defina
$Z=F_{X}(X)$. Mostre que $Z\sim U(0,1)$}}


\subsubsection*{\textmd{Resp. b) Para resolver esse problema, basta mostrarmos que
$f_{Z}(z)=1$. A informação relevante aqui é que $F_{X}$ é estritamente
monótona, o que impliqua que é contínua e que possui inversa denotada
por $F_{x}^{-1}$ que também é contínua. Disso temos que $P(Z\leq z)=P(F_{X}(X)\leq z)=P(F_{X}^{-1}(F_{X}(X))\leq F_{X}^{-1}(z))=P(X\leq F_{X}^{-1}(z))=P(X\leq F_{X}^{-1}(z))=F_{X}(F_{X}^{-1}(z))=z$.
Como $F_{z}(z)=P(Z\leq z)=z$ temos que $f_{Z}(z)=\frac{dF_{z}(z)}{dz}=1$.}}


\subsubsection*{\textmd{c) Tome $U\sim U(0,1)$ e mostre que $W=F_{X}^{-1}(U)$ tem
f.d.a. $F_{X}$.}}


\subsubsection*{\textmd{Resp. c) Análogo a b)$F_{W}(w)=P(W\leq w)=P(F_{X}^{-1}(U)\leq w)=P(F_{X}(F_{X}^{-1}(U))\leq F_{X}(w))=P(U\leq F_{X}(w))=\int_{0}^{F_{X}(w)}1du=F_{x}(w)$.
Observe que a última integral é válida pois $0\leq F_{X}(w)\leq1$,
senão teríamos que avaliar intervalos de restrição.}}


\subsection*{\textcompwordmark{}}


\subsection*{\textmd{10) Seja $U\sim Uniforme(0,1)$ e $X=ln(U)$. Ache a densidade
e a função geratriz de momentos de X. Usando a função geratriz de
momentos, calcule $\mathbf{E}(X)$ e $Var(X)$.}}


\subsubsection*{\textmd{Resp.) $F_{X}(x)=P(X\leq x)=P(ln(U)\leq x)=P(U\leq e^{x})=F_{U}(e^{x})\protect\overset{11}{=}\protect\begin{cases}
\protect\begin{array}{c}
e^{x}\protect\\
1
\protect\end{array} & \protect\begin{array}{c}
0\leq e^{x}\leq1\protect\\
e^{x}>1
\protect\end{array}\protect\end{cases}\protect\implies f_{x}(x)=\frac{dF_{X}(x)}{dx}=\protect\begin{cases}
\protect\begin{array}{c}
e^{x}\protect\\
0
\protect\end{array} & \protect\begin{array}{c}
x\leq0\protect\\
x>0
\protect\end{array}\protect\end{cases}$}}
\begin{itemize}
\item Observe que $0\leq e^{x}\leq1\Longleftrightarrow-\infty<x\leq0$
\end{itemize}

\subsubsection*{\textmd{Com isso, podemos calcular $M_{X}(t)=\int_{-\infty}^{\infty}e^{t\cdot x}\cdot f_{X}(x)dx=\int_{-\infty}^{0}e^{t\cdot x}\cdot e^{x}dx=\int_{-\infty}^{0}e^{(t+1)\cdot x}dx=\frac{1}{t+1}$}}


\subsubsection*{\textmd{E com isso $\mathbf{E}(X)=\frac{dM_{X}(t)}{dx}=-\frac{1}{(t+1)^{2}}_{t=0}=-1$}}


\subsubsection*{\textmd{E $Var(X)=\frac{d^{2}M_{X}(t)}{dx^{2}}=\frac{2}{(t+1)^{3}}_{t=0}=1$}}


\subsection*{\textmd{11) Para quaisquer eventos A, B e C mostre que:}}


\subsubsection*{\textmd{a) se $A\subset B$, então $B^{c}\subset A^{c}$}}

Resp. a) Vou usar o conhecimento prévio que $A\subset B\Longleftrightarrow A\cup C\subset B\cup C$
para qualquer conjunto $C.$ E que $A\subset B$ pelo enunciado

$\Omega=B^{c}\cup B=A\cup A^{c}\subset B\cup A^{c}\implies B^{c}\cup B\subset A^{c}\cup B\implies B^{c}\subset A^{c}$


\subsubsection*{\textmd{b) $(A\bigcup B)^{c}$ $A^{c}\bigcap B^{c}$ }}

Resp. b) Vou usar algum conhecimento anterior também:

0)$A\cap B\subset A\cup B$

i)$A\subset B\implies B^{c}\subset A^{c}$

ii)$A\subset C$ e $B\subset C\implies A\cup B\subset C$

iii)$A\subset B$ e $A\subset C\implies A\subset C\cap B$

Dessas relações conseguimos:

iv)$B^{c}\cap A^{c}\subset B^{c}\overset{i}{\implies}B\subset(A^{c}\cap B^{c})^{c}$

v)$B^{c}\cap A^{c}\subset A^{c}\overset{i}{\implies}A\subset(A^{c}\cap B^{c})^{c}$

vi)$B\subset A\cup B\overset{i}{\implies}(A\cup B)^{c}\subset B^{c}$

vii)$A\subset A\cup B\overset{i}{\implies}(A\cup B)^{c}\subset A^{c}$

viii) usando 0, ii, iv, v temos: $A\cap B\subset A\cup B\subset(A^{c}\cap B^{c})^{c}\overset{i}{\implies}A^{c}\cap B^{c}\subset(A\cap B)^{c}$

ix) Análogo, usando iii, vi, vii temos: $(A\cup B)^{c}\subset A^{c}\cap B^{c}$

Finalmente, como de viii e ix temos que $A^{c}\cap B^{c}\subset(A\cap B)^{c}$
e $(A\cup B)^{c}\subset A^{c}\cap B^{c}$ então chegamos à conclusão
que $(A\cup B)^{c}=A^{c}\cap B^{c}$


\subsubsection*{\textmd{c) $A\bigcap(B\bigcup C)=(A\bigcap B)\bigcup(A\bigcap C)$}}

Resp. c) Como não podemos usar para a prova essa relação que é uma
das mais úteis em teoria dos conjuntos, vamos ter que usar uma carga
pesada de conhecimento anterior (para economizar espaço utilizarei
a notação $AB\equiv A\cap B$):

a)$A\subset B\Longleftrightarrow AC\subset BC$

b)$A=AB^{c}\overset{.}{\cup}AB$

c)$AA=A$

d)$A\subset\Omega$

e)$\Omega A=A$

f)$A\subset B\Longleftrightarrow B^{c}\subset A^{c}$

g)$A\subset B\Longleftrightarrow C\cup A\subset C\cup B$

h)$AA^{c}=\emptyset$

i)$A\subset B$ e $C\subset D\implies AC\subset BD$ e $A\cup C\subset B\cup D$

j)$AB\subset A$

k)$(A\cup B)^{c}=A^{c}B^{c}$

l)$A\subset A\cup B$

Com isso podemos atacar o problema. Vamos começar tentando encontrar
uma relação tal que $A\bigcap(B\bigcup C)\subset(A\bigcap B)\bigcup(A\bigcap C)$:

1) Considere o conjunto $(A^{c}\cup B^{c})\cap(A^{c}\cup C^{c})$

1.i) Vamos quebrar o problema em partes menores, aqui vamos mostrar
que $(A^{c}\cup B^{c})\cap A\subset B^{c}A$:

$(A^{c}\cup B^{c})\cap A\overset{b,h}{=}([A^{c}B]\overset{.}{\cup}[A^{c}B^{c}]\overset{.}{\cup}[AB^{c}])\cap A\Longleftrightarrow\begin{cases}
\begin{array}{c}
x\in A{}^{c}B\implies A^{c}BA\overset{h}{=}\emptyset\\
x\in A{}^{c}B^{c}\implies A^{c}B^{c}A\overset{h}{=}\emptyset\\
x\in AB^{c}\implies AB^{c}A\overset{c}{=}B^{c}A
\end{array}\end{cases}$

Disso concluímos que $(A^{c}\cup B^{c})\cap A\subset B^{c}A$

1.ii) Na mesma linha, vamos mostrar que $(A^{c}\cup C^{c})\cap A\subset C^{c}A$:

Vamos reescrever $(A^{c}\cup C^{c})\cap A$ como uma união disjunta
interseccionada com $A$

$(A^{c}\cup C^{c})\cap A\overset{b,h}{=}([A^{c}C]\overset{.}{\cup}[A^{c}C^{c}]\overset{.}{\cup}[AC^{c}])\cap A\Longleftrightarrow\begin{cases}
\begin{array}{c}
x\in A{}^{c}C\implies A^{c}CA\overset{h}{=}\emptyset\\
x\in A{}^{c}C^{c}\implies A^{c}C^{c}A\overset{h}{=}\emptyset\\
x\in AC^{c}\implies AC^{c}A\overset{c}{=}C^{c}A
\end{array}\end{cases}$

Disso concluímos que $(A^{c}\cup C^{c})\cap A\subset C^{c}A$

1.1) A partir de 1.i e 1.ii e usando a, c, i temos que $(A^{c}\cup C^{c})\cap A\subset C^{c}A$
e $(A^{c}\cup B^{c})\cap A\subset B^{c}A\implies(A^{c}\cup B^{c})\cap(A^{c}\cup C^{c})\cap A\subset B^{c}C^{c}A\overset{a}{\implies}(A^{c}\cup B^{c})\cap(A^{c}\cup C^{c})\subset B^{c}C^{c}$

1.2) Agora vamos provar a relação com o complementar de $A$:

$(A^{c}\cup B^{c})\cap(A^{c}\cup C^{c})\cap A^{c}\overset{c}{\subset}[(A^{c}\cup B^{c})\cap A^{c}]\cap[(A^{c}\cup C^{c})\cap A^{c}]\overset{j}{\subset}A^{c}$ 

1.final) De 1.1, 1.2, g, l temos que $(A^{c}\cup B^{c})\cap(A^{c}\cup C^{c})\subset[(A^{c}\cup B^{c})\cap(A^{c}\cup C^{c})\cap A^{c}]\overset{.}{\cup}[(A^{c}\cup B^{c})\cap(A^{c}\cup C^{c})]\subset A^{c}\cup B^{c}C^{c}$
logo

$(A^{c}\cup B^{c})\cap(A^{c}\cup C^{c})\subset A^{c}\cup B^{c}C^{c}\overset{k,f}{\implies}A\cap(B\cup C)\subset(A\cap B)\cup(A\cap C)$

2) Agora basta provar a volta $(A\cap B)\cup(A\cap C)\subset A\cap(B\cup C)$
esta etapa é relativamente menos custosa que a primeira:

2.1) $B\subset B\cup C\overset{a}{\implies}AB\subset A\cap(B\cup C)$

2.2) $C\subset B\cup C\overset{a}{\implies}AC\subset A\cap(B\cup C)$

2.final) De 2.1 e 2.2 e utilizando i temos que $AB\cup AC\subset A\cap(B\cup C)$

Resp. c final) de 1.final e 2.final temos que $AB\cup AC=A\cap(B\cup C)$

\textbf{\textcompwordmark{}}


\subsection*{\textmd{12) Numa urna há 5 bolinhas brancas, 4 verdes e 6 azuis.
Escolhemos 4 bolinhas. Qual é a probabilidade de que foram escolhidas
2 bolinhas de uma cor e 2 bolinhas de outra cor? Qual é a probabilidade
de que todas as bolinhas escolhidas são da mesma cor? Considere dois
casos: escolha sem reposição e com reposição.}}


\subsubsection*{\textmd{Resp. Sem reposição)}}

Temos que o conjunto de todas as combinações possiveis dado por $\Omega$
tem $\#\Omega=\left(\begin{array}{c}
15\\
4
\end{array}\right)$ possibilidades

Seja o conjunto $A$:''retirar duas bolas de uma cor e duas de outra
cor'', com isso $\#A=\left(\begin{array}{c}
5\\
2
\end{array}\right)\cdot\left(\begin{array}{c}
4\\
2
\end{array}\right)+\left(\begin{array}{c}
5\\
2
\end{array}\right)\cdot\left(\begin{array}{c}
6\\
2
\end{array}\right)+\left(\begin{array}{c}
4\\
2
\end{array}\right)\cdot\left(\begin{array}{c}
6\\
2
\end{array}\right)$

Logo, $P(A)=\frac{\#A}{\#\Omega}=\frac{\left(\begin{array}{c}
5\\
2
\end{array}\right)\cdot\left(\begin{array}{c}
4\\
2
\end{array}\right)+\left(\begin{array}{c}
5\\
2
\end{array}\right)\cdot\left(\begin{array}{c}
6\\
2
\end{array}\right)+\left(\begin{array}{c}
4\\
2
\end{array}\right)\cdot\left(\begin{array}{c}
6\\
2
\end{array}\right)}{\left(\begin{array}{c}
15\\
4
\end{array}\right)}$

Considere o conjunto $B$:''retirar quatro vezes a mesma cor''

$\#B=\left(\begin{array}{c}
5\\
4
\end{array}\right)+\left(\begin{array}{c}
4\\
4
\end{array}\right)+\left(\begin{array}{c}
6\\
4
\end{array}\right)$

Logo, $P(B)=\frac{\#B}{\#\Omega}=\frac{\left(\begin{array}{c}
5\\
4
\end{array}\right)+\left(\begin{array}{c}
4\\
4
\end{array}\right)+\left(\begin{array}{c}
6\\
4
\end{array}\right)}{\left(\begin{array}{c}
15\\
4
\end{array}\right)}$


\subsubsection*{\textmd{Resp. Com reposição)}}

Temos que o conjunto de todas as combinações possiveis dado por $\Omega$
tem $\#\Omega=15^{4}$ possibilidades

Seja o conjunto $A$:''retirar duas bolas de uma cor e duas de outra
cor'', com isso $\#A=5\cdot5\cdot4\cdot4+5\cdot5\cdot6\cdot6+4\cdot4\cdot6\cdot6$

Logo, $P(A)=\frac{\#A}{\#\Omega}=\frac{5\cdot5\cdot4\cdot4+5\cdot5\cdot6\cdot6+4\cdot4\cdot6\cdot6}{15^{4}}=\frac{5^{2}\cdot4^{2}+5^{2}\cdot6^{2}+4^{2}\cdot6^{2}}{15^{4}}$

Considere o conjunto $B$:''retirar quatro vezes a mesma cor''

$\#B=5.5.5.5+6.6.6.6+4.4.4.4$

Logo, $P(B)=\frac{\#B}{\#\Omega}=\frac{5.5.5.5+6.6.6.6+4.4.4.4}{15^{4}}=\frac{5^{4}+6^{4}+4^{4}}{15^{4}}$

\textbf{\textcompwordmark{}}


\subsection*{\textmd{13) Seja X uma variável aleatória discreta com $P(X=0)=0.25$,
$P(X=1)=0.125$, $P(X=2)=0.125$, $P(X=3)=0.5$. Calcule a função
de distribuiçãoo acumulada, o valor esperado e a variância de X. Determine
as seguintes probabilidades: $P(0<X<1)$, $P(X\le1)$, $P(X>2)$,
$P(X>2.5)$.}}


\subsubsection*{\textmd{Resp. )}}

$F_{X}(x)=P(X\leq x)=\sum_{i=-\infty}^{i=x}P(X=i)=\begin{cases}
0 & se\: x<0\\
0,25 & se\:0\leq x<1\\
0,375 & se\:1\leq x<2\\
0,5 & se\:2\leq x<3\\
1 & se\: x\geq3
\end{cases}$

Com essa definição temos:

$P(0<X<1)=0,25-0,25=0$

$P(X\le1)=0,375$

$P(X>2)=1-F_{X}(2)=0,5$

$P(X>2.5)=1-F_{X}(2,5)=0,5$

Para calcular a Esperança, utilizamos a definição

$\mathbf{E}(X)=\sum_{x\in X}[P(X=x)\cdot x]=0,125+2\cdot0,125+3\cdot0,5=2,125$

Para calcular a Variância, usamos a esperança de $X^{2}$

$\mathbf{E}(X^{2})=\sum_{x\in X}[P(X=x)\cdot x^{2}]=0,125+4\cdot0,125+9\cdot0,5=5,625$

$Var(X)=\mathbf{E}(X^{2})-\mathbf{E}(X)^{2}=5,625-(2,125)^{2}$

\textbf{\textcompwordmark{}}


\subsection*{\textmd{14) Suponha que o tempo de viagem entre sua casa e UNICAMP
tem distribuição Normal com média 50 minutos e desvio padrão 4 minutos.
Se você tem uma prova as 10:00 e quer que probabilidade de chegar
atrasado seja no máximo 0.5\%, a que horas você deve sair de casa?}}


\subsubsection*{\textmd{Resp. ) }}

O tempo que demoramos é uma v.a. normal $T\sim N(50,16)$, já que
$\sigma=4\implies\sigma^{2}=16$.

Seja $H$ a hora que temos que acordar:

$P(H+T>10\cdot60)\leq0,005\implies P(H>600-T)\leq0,005\implies P(\frac{H-50}{4}>\frac{600-T-50}{4})\leq0,005\implies1-\phi(\frac{550-H}{4})\leq0,005\implies\phi(\frac{550-H}{4})\geq0,995$

Olhando na tabela de distribuição normal, vemos que o valor para que
isso ocorra é $2,58$ daí:

$\frac{550-H}{4}=2,58\implies H=539,68$, assim temos que acordar
no máximo às 8:59:41

\textbf{\textcompwordmark{}}\\

\begin{quotation}
Este solucionário foi feito para a disciplina ME310 - 2Sem 2012. Caso
encontre algum erro, por favor peça alteração informando o erro em
nosso grupo de discussão: 

$$https://groups.google.com/forum/?fromgroups\#!forum/me310-2s-2012$$

ou diretamente no repositório do github:

$$https://github.com/nullhack/Probabilidade2$$

Bons estudos,

Eric.\end{quotation}

\end{document}
